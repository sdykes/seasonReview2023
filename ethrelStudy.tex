% Options for packages loaded elsewhere
\PassOptionsToPackage{unicode}{hyperref}
\PassOptionsToPackage{hyphens}{url}
\PassOptionsToPackage{dvipsnames,svgnames,x11names}{xcolor}
%
\documentclass[
  letterpaper,
  DIV=11,
  numbers=noendperiod]{scrartcl}

\usepackage{amsmath,amssymb}
\usepackage{iftex}
\ifPDFTeX
  \usepackage[T1]{fontenc}
  \usepackage[utf8]{inputenc}
  \usepackage{textcomp} % provide euro and other symbols
\else % if luatex or xetex
  \usepackage{unicode-math}
  \defaultfontfeatures{Scale=MatchLowercase}
  \defaultfontfeatures[\rmfamily]{Ligatures=TeX,Scale=1}
\fi
\usepackage{lmodern}
\ifPDFTeX\else  
    % xetex/luatex font selection
\fi
% Use upquote if available, for straight quotes in verbatim environments
\IfFileExists{upquote.sty}{\usepackage{upquote}}{}
\IfFileExists{microtype.sty}{% use microtype if available
  \usepackage[]{microtype}
  \UseMicrotypeSet[protrusion]{basicmath} % disable protrusion for tt fonts
}{}
\makeatletter
\@ifundefined{KOMAClassName}{% if non-KOMA class
  \IfFileExists{parskip.sty}{%
    \usepackage{parskip}
  }{% else
    \setlength{\parindent}{0pt}
    \setlength{\parskip}{6pt plus 2pt minus 1pt}}
}{% if KOMA class
  \KOMAoptions{parskip=half}}
\makeatother
\usepackage{xcolor}
\setlength{\emergencystretch}{3em} % prevent overfull lines
\setcounter{secnumdepth}{-\maxdimen} % remove section numbering
% Make \paragraph and \subparagraph free-standing
\ifx\paragraph\undefined\else
  \let\oldparagraph\paragraph
  \renewcommand{\paragraph}[1]{\oldparagraph{#1}\mbox{}}
\fi
\ifx\subparagraph\undefined\else
  \let\oldsubparagraph\subparagraph
  \renewcommand{\subparagraph}[1]{\oldsubparagraph{#1}\mbox{}}
\fi


\providecommand{\tightlist}{%
  \setlength{\itemsep}{0pt}\setlength{\parskip}{0pt}}\usepackage{longtable,booktabs,array}
\usepackage{calc} % for calculating minipage widths
% Correct order of tables after \paragraph or \subparagraph
\usepackage{etoolbox}
\makeatletter
\patchcmd\longtable{\par}{\if@noskipsec\mbox{}\fi\par}{}{}
\makeatother
% Allow footnotes in longtable head/foot
\IfFileExists{footnotehyper.sty}{\usepackage{footnotehyper}}{\usepackage{footnote}}
\makesavenoteenv{longtable}
\usepackage{graphicx}
\makeatletter
\def\maxwidth{\ifdim\Gin@nat@width>\linewidth\linewidth\else\Gin@nat@width\fi}
\def\maxheight{\ifdim\Gin@nat@height>\textheight\textheight\else\Gin@nat@height\fi}
\makeatother
% Scale images if necessary, so that they will not overflow the page
% margins by default, and it is still possible to overwrite the defaults
% using explicit options in \includegraphics[width, height, ...]{}
\setkeys{Gin}{width=\maxwidth,height=\maxheight,keepaspectratio}
% Set default figure placement to htbp
\makeatletter
\def\fps@figure{htbp}
\makeatother

\usepackage{booktabs}
\usepackage{longtable}
\usepackage{array}
\usepackage{multirow}
\usepackage{wrapfig}
\usepackage{float}
\usepackage{colortbl}
\usepackage{pdflscape}
\usepackage{tabu}
\usepackage{threeparttable}
\usepackage{threeparttablex}
\usepackage[normalem]{ulem}
\usepackage{makecell}
\usepackage{xcolor}
\KOMAoption{captions}{tableheading}
\makeatletter
\makeatother
\makeatletter
\makeatother
\makeatletter
\@ifpackageloaded{caption}{}{\usepackage{caption}}
\AtBeginDocument{%
\ifdefined\contentsname
  \renewcommand*\contentsname{Table of contents}
\else
  \newcommand\contentsname{Table of contents}
\fi
\ifdefined\listfigurename
  \renewcommand*\listfigurename{List of Figures}
\else
  \newcommand\listfigurename{List of Figures}
\fi
\ifdefined\listtablename
  \renewcommand*\listtablename{List of Tables}
\else
  \newcommand\listtablename{List of Tables}
\fi
\ifdefined\figurename
  \renewcommand*\figurename{Figure}
\else
  \newcommand\figurename{Figure}
\fi
\ifdefined\tablename
  \renewcommand*\tablename{Table}
\else
  \newcommand\tablename{Table}
\fi
}
\@ifpackageloaded{float}{}{\usepackage{float}}
\floatstyle{ruled}
\@ifundefined{c@chapter}{\newfloat{codelisting}{h}{lop}}{\newfloat{codelisting}{h}{lop}[chapter]}
\floatname{codelisting}{Listing}
\newcommand*\listoflistings{\listof{codelisting}{List of Listings}}
\makeatother
\makeatletter
\@ifpackageloaded{caption}{}{\usepackage{caption}}
\@ifpackageloaded{subcaption}{}{\usepackage{subcaption}}
\makeatother
\makeatletter
\@ifpackageloaded{tcolorbox}{}{\usepackage[skins,breakable]{tcolorbox}}
\makeatother
\makeatletter
\@ifundefined{shadecolor}{\definecolor{shadecolor}{rgb}{.97, .97, .97}}
\makeatother
\makeatletter
\makeatother
\makeatletter
\makeatother
\ifLuaTeX
  \usepackage{selnolig}  % disable illegal ligatures
\fi
\IfFileExists{bookmark.sty}{\usepackage{bookmark}}{\usepackage{hyperref}}
\IfFileExists{xurl.sty}{\usepackage{xurl}}{} % add URL line breaks if available
\urlstyle{same} % disable monospaced font for URLs
\hypersetup{
  colorlinks=true,
  linkcolor={blue},
  filecolor={Maroon},
  citecolor={Blue},
  urlcolor={Blue},
  pdfcreator={LaTeX via pandoc}}

\author{}
\date{}

\begin{document}
\ifdefined\Shaded\renewenvironment{Shaded}{\begin{tcolorbox}[sharp corners, borderline west={3pt}{0pt}{shadecolor}, boxrule=0pt, enhanced, breakable, interior hidden, frame hidden]}{\end{tcolorbox}}\fi

\# Ethephon spray applications on PremA96 apples 2023 \{\#sec-ethrel\}

Authors: Niemann N1, Marinkovich D2, Julian A1, Palmer G1, Mair S1,
Johnston J1

\hypertarget{summary}{%
\subsection{Summary}\label{summary}}

Remarkable growth in Malus domestica `PremA96' plantings in New Zealand
is leading to logistics pressures at storage facilities and the
packhouse. Multiple actions can be brought into practice to streamline
and ease these pressures, including using growth regulators such as the
ethylene-producing compound ethephon (which triggers maturation,
ripening and senescence in fruit) and the compound that inhibits
ethylene's action, 1-methylcyclopropene (1-MCP). Both these compounds
can be applied in the orchard as spray treatments, and both have
physiological effects on the fruit that may continue into storage.
Ethephon spray treatments (Ethin™, containing 480 g/L chlorethephon) may
bring the start of harvesting forward by a few weeks, while the research
last year on the spray application of 1-MCP-containing Harvista™ 1.3 SC
proved it could delay the end of the harvest period by up to 4 weeks.

Ethephon treatments at two doses (applied at 200 and 500 mL/ha) were
applied two or three times in test blocks on `PremA96' trees in January
2023. Fruit maturity was tracked on a weekly basis to determine optimum
harvest windows and to follow how fruit quality changed. One 200 mL/ha
block furthermore received two Harvista spray treatments to determine
how 1-MCP treatment can influence Ethephon treatment. Apples were
harvested weekly from 11 January to 4 April 2023 and placed in regular
air (RA) or controlled atmosphere (CA) storage after a SmartFresh™
treatment for 3 and 6 months to determine how these growth regulators
affect storage performance of the apples.

The key findings of this trial were:

\begin{itemize}
\tightlist
\item
  Ethephon treatment brought the recommended harvest date forward from
  21 February to as early as 30 January, depending on application number
  and dose. The harvest window (when average starch pattern index (SPI)
  was 3--4) was longer for Ethephon-treated fruit compared with
  unsprayed fruit. The timing of application may be more important than
  the number of spray applications, since little difference could be
  measured between blocks that received two and three applications.
\item
  The lower, label-recommended dose of 200 g/ha Ethin moved the
  recommended start date of harvest forward by only a week relative to
  untreated controls, and there was considerable overlap with the time
  when unsprayed fruit could be picked. It is unclear whether this dose
  can have a stronger effect if applied earlier.
\item
  Following Ethephon treatments with Harvista sprays can slow ripening
  down again, therefore fruit could be picked from the same time and up
  to 2 weeks later than the control fruit (21 February to 14 March).
  These fruit stored better than just Ethephon-treated fruit.
\item
  Ethephon treatments resulted in fruit with different traits than
  unsprayed fruit when their SPI was 3--4. Fruit firmness was higher
  than control fruit during their recommended harvest times. Dry matter
  content was lower in Ethephon-treated fruit, but soluble solids
  concentration (SSC) was similar. The implication is that earlier
  harvests deprive fruit of the opportunity to accumulate structural
  carbon, although SSC is similar between control and Ethephon-treated
  fruit.
\item
  Fruit permeance decreased as the fruit matured. Picking any fruit
  earlier (Ethephon-treated at correct maturity, or immature unsprayed
  fruit) increase the risk of shrivel symptom development in storage.
\item
  Storage tests revealed that Ethephon does have a negative effect on
  the times fruit can be stored. The most important quality parameter
  affected was shrivel symptom development. Up to 30\% more
  Ethephon-treated fruit would develop shrivel compared with unsprayed
  fruit in the same storage environment.
\item
  Internal ethylene concentration (IEC) and SPI changes correlated with
  each other, proving that SPI alone can be used to judge when fruit are
  ready for picking.
\end{itemize}

\hypertarget{introduction}{%
\subsection{Introduction}\label{introduction}}

Ethephon (2-chloroethylphosphonic acid) is one of the most commonly used
plant growth regulators that can promote pre- and postharvest ripening.
After being absorbed by the plant, the molecule is metabolised and
ethylene is released. In addition to ripening, ethylene can induce
abscission, flower induction and leaf senescence
{[}@wang\_aminoethoxyvinylglycine\_2001{]}. In order to limit these
effects, Ethephon is often applied in conjunction with the synthetic
auxin plant hormone naphthalene acetic acid (NAA). NAA is known to
prevent premature fruit dropping by inhibiting abscission. It can help
to limit leaf drop as ethylene concentrations increase in and around the
trees after Ethephon application, but likewise accelerates ripening in
apples {[}@ozturk\_effects\_2019{]}. Additionally, it may exacerbate
russet in susceptible apple cultivars {[}@jones\_reservations\_1991{]}.

The benefits of exposing apples (Malus domestica) to ethylene during the
later stages of their development include improved colour development
and being ready for harvest earlier. This could help to ease picking
pressures during the harvest window. But once ripening has sped up,
senescence will follow. The implications of this include the fact that
the apples may not be stored as long as otherwise could be expected,
with storage and aging disorders more likely to develop.
1-Methylcyclopropene (1-MCP) technologies such as SmartFresh™ and
Harvista™ could help to slow senescence again, but since the processes
have already started in the orchard, these mitigations will have limited
effects {[}@mair\_s\_effect\_2021{]}.

The aim of this project was to track fruit ripening in the orchard for a
period after Ethephon spray applications, and to determine how well the
fruit stored after the various treatments. Multiple application regimes
were tracked, with two dosage concentrations applied to different blocks
in the same orchard. In addition, one block received a Harvista spray
treatment after conventional Ethephon treatment to investigate how 1-MCP
can block the cascade of effects that Ethephon may have on `PremA96'
apples.

\hypertarget{materials-methods}{%
\subsection{Materials \& methods}\label{materials-methods}}

\hypertarget{spray-application}{%
\subsubsection{Spray application}\label{spray-application}}

A Rockit\textsuperscript{TM} apple orchard was selected for this trial,
with several blocks receiving different spray treatments to address the
questions in this trial. The spray treatments and dates are recorded in
Table 1. All blocks received Erger (Valagro®) treatment to stimulate bud
break in August 2022, but different doses were applied to some blocks
since the experimental plans for this trial were not in place yet.

The block reserved for control fruit sampling was accidentally picked by
orchard staff towards the end of January, necessitating the use of a
second block (designated new control) to sample fruit for the remainder
of the trial.

\begin{table}[H]
\centering
\begin{tabular}{l|l|r|l}
\hline
block & date & days after first Ethephon spray & spray details\\
\hline
\makecell[l]{E500 x 3\\E500 x 2\\ E200 x 2\\Control} & 8th August 2022 & NA & 5 L/100 L Erger (Valagro®)\\
\hline
\makecell[c]{New Control E200\\+ Harvista} & 21st August 2022 & NA & 2.5 L/100 L Erger (Valagro®)\\
\hline
E500 x 3 & 11th January 2023 & -1 & 500 mL Ethin/ha, NAA 100 (10%), Spray Aid\\
\hline
\makecell[l]{E500 x 3\\E500 x 2} & 21st January 2023 & 9 & 500 mL Ethin/ha, NAA 100 (10%), Spray Aid\\
\hline
\makecell[c]{E200 x 2\\E200 + Harvista} & 27th January 2023 & 15 & Captan 80 WG, 200 mL Ethin/ha, NAA 100 (10%), Spray Aid\\
\hline
E200 + Harvista & 2nd february 2023 & 21 & 200 mL Ethin/ha, Regulaid, NAA 100 (10%), Spray Aid\\
\hline
E500 x 3 & 2nd February 2023 & 21 & 500 mL Ethin/ha, NAA 100 (10%), Spray Aid\\
\hline
E500 x 2 & 9th February 2023 & 28 & Captan 80 WG, 200 mL Ethin/ha, NAA 100 (10%), Spray Aid\\
\hline
E200 + Harvista & 10th February 2023 & 29 & Harvista (150 g/ha)\\
\hline
E200 + Harvista & 22nd February 2023 & 41 & Harvista (150 g/ha)\\
\hline
\end{tabular}
\end{table}



\end{document}
